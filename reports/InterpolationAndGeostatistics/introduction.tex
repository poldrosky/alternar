\chapter*{INTRODUCCIÓN}
\addcontentsline{toc}{chapter}{INTRODUCCIÓN}
 
Datos geoestadísticos son datos que podrían en principio ser medidos en cualquier lugar,
pero que normalmente vienen como mediciones en un número limitado de ubicaciones de observación: piensa en la calidad de oro en un yacimiento o partículas en muestras
de aires. El patrón de ubicaciones de observación general, no es de interés primordial, ya que a menudo resulta de consideraciones que van desde limitaciones
económicas y físicas a ser 'representativos' o variedades de muestreo aleatorios.
El interés es por lo general en la inferencia de los aspectos de la variable que no tiene medidas como mapas de los valores estimados,
probabilidades de excedencia o estimaciones de los agregados más regiones indicadas o inferencia del proceso que generaron los datos.
Otros problemas son la supervisión de la red de optimización: ¿Dónde deben ubicarse las nuevas observaciones o cuales ubicaciones de observación
deben ser retirados de manera que el valor operacional de la vigilancia la red se maximiza.

Problemas espaciales típicas donde se utilizan geoestadística son los siguientes:
\begin{itemize}
\item La estimación de las calidades de unidades de minerales explotables, basado en datos de perforación.
\item  La interpolación de variables ambientales de la muestra o de la red de monitoreo de datos
  (por ejemplo, la calidad del aire, la contaminación del suelo, la cabeza del agua subterránea, conductividad hidráulica).
\item La interpolación de variables físicas o químicas a partir de datos de la muestra.
\item La estimación de promedios espaciales a partir de datos continuos, espacialmente correlacionadas
\end{itemize}


\nocite{bivand2013applied}

