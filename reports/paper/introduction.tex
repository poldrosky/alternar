\section{Introducción}

\IEEEPARstart{E}{l} proyecto general ``Análisis de oportunidades energéticas con fuentes
alternativas en el departamento de Nariño'' se ha planteado como uno de sus
objetivos la medición y estimación de potenciales energéticos en zonas viables
de la región. Uno de los componentes a analizar es el potencial de biomasa para
la generación eléctrica. Uno de los problemas que se plantea para la ubicación
de lugares propicios es la ausencia de bases de datos históricas en el área de
estudio que permitan su respectivo análisis.
Diversas investigaciones han demostrado la utilidad del uso de imágenes
satelitales para la generación de modelos que permitan calcular la cantidad de
biomasa presente en un determinado lugar. En la actualidad, se cuenta con
libre acceso al repositorio de imágenes satelitales Landsat que con el debido
tratamiento pueden ser usadas para calcular valores nominales de biomasa. Sin
embargo, dichos modelos requieren la ejecución de trabajo de campo en la zona
para inferir fórmulas iniciales a partir de la medición tradicional de una muestra.
Esta investigación esta orientada a cumplir con los requerimientos necesarios
para la generación de un modelo de predicción de biomasa y su extrapolación
al resto del área de estudio.
