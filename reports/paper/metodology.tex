 
\section{Metodología}

\subsection{Obtención de datos}

El proceso de obtención de datos se lo realizó tomando imágenes satelitales que provee el satélite
Landsat 7. En este proceso se descargaron 1362 imágenes satelitales desde el año 1999 hasta el año 2015, que cubren el 
departamento de Nariño. Para cubrir todo el departamento fue necesario descargar las imagenes satelitales con 
los siguientes paths y rows: (009,059), (009,060), (010,058), (010,059), (011,059) 

En la obtención de datos también se utilizó el mapa de biomass construido por \cite{baccini2012estimated} 
construido para el año 2000 a 2003.


\subsection{Preprocesamiento}

En esta etapa de preprocesamiento se reproyecto las imágenes obtenidas, debido a que las cinco imágenes
que cubren el departamento de Nariño, estan en distintos sistemas de coordenadas (EPSG:32618 y EPSG:32617) y se 
lo reproyecto al sitema EPSG:3857. Así como también se recorto las imágenes con el fin de unicamente tener 
el área que cubre el departamento de Nariño, como lo muestra la figura~\ref{fig:Recortar imágenes}

\begin{figure}
  \centering
  \subfigure[Imágenes Satélitales de Nariño]{\label{Imágenes Satélitales Nariño} \includegraphics[width= 5cm]{cut1.png}}
  \vfill
  \subfigure[Imágenes recortadas de Nariño]{\label{Imágenes recortadas de Nariño}\includegraphics[width= 5cm]{cut2.png}}
  \caption{Prepocesamiento}
  \label{fig:Recortar imágenes}
\end{figure}


De igual manera este proceso se lo realizó para el mapa de biomasa, como se muestra en la figura~\ref{fig:mapaNarino}


\begin{figure}
  \centering
  \includegraphics[width = 5cm]{mapaNarino.png}
  \caption{Mapa de Nariño .}
  \label{fig:mapaNarino}
\end{figure}

\subsection{Procesamiento y limpieza de datos}

Para esta etapa, primero se diseño una base de datos para capturar los datos,
como lo muestra la figura~\ref{fig:landsatET}, la cual tiene 4 tablas. 

\begin{figure}
  \centering
  \includegraphics[width = 8cm]{landsatET.pdf}
  \caption{Modelo entidad-relacion Landsat}
  \label{fig:landsatET}
\end{figure}

Tabla date\_landsat: en la cual se almacenan las fechas de las imágenes satelitales.

Tabla reflectance: en la cual se almacenan los datos capturados y convertidos en reflentance,
de las bandas landsat (1 - 5,7) y la temperatura en grados kelvin de la banda 6.

Tabla discarded: en la cual se almacenan datos que fueron descartados, por varias rasones,
son nubes calientes, nubes frias, datos ambiguos o no son vegetación.

Tabla biomass: en la cual se almacenan los datos de biomassa del mapa de \cite{baccini2012estimated}.

Para procesar las imágenes y llenar la base de datos se realizó un Script, el cual captura el Digital Number
de las imágenes satélitales y lo transforma en valor en reflectance. En este procesamiento de imagenes, se adiciono al Script unos filtros para para detección de nubes calientes,
nubes, frias, datos ambiguos como lo muestra el algoritmo propuesto por \cite{irish2000landsat}, además se aplico 
un filtro adicional, el NVDI(normalized difference vegetation index) para trabajar unicamente con datos de vegetación.

\subsection{Análisis de regresión}

El análisis de regresión se lo realizó iterando los valores de las bandas agrupados en cada punto entre el año 2000 y 2003 datos con los que se realiza la
regresión frente al valor de biomasa obtenido en \cite{baccini2012estimated}, y asi tener un promedio con más muestras
en cada punto y obtener el mejor modelo, esta iteración se la realizó desde 1 muestra hasta 45 muestras, obteniendo el mejor modelo con 35 muestras
las cuales tienen 1009 datos. Con menos muestras hay más datos y eso hace que no se encuentre un buen modelo, pero cuando hay mas muestras los datos son menores
y esto también hace que el resultado del modelo tampoco sea bueno.

En la tabla~\ref{tab:metricas} se muestra las métricas de los modelos analizadas con 35 muestras y 1009 datos, que es el modelo escogido, esta tabla 
se la realizó usando el paquete de mineria de datos rminer.

\begin{table}
\caption{Métricas de modelos analizados con 35 muestras y 1009 datos}
\label{tab:metricas}
\centering
\scalebox{0.7}{
\begin{tabular}{c c c c c c c}
\toprule
 & SAE& MAE & RAE & RMSE & COR & R2 \\
\midrule
ctree & 10406.58225 & 30.88007 & 65.04650 & 40.02893 & 0.69401 & 0.48165 \\
rpart & 10197.95826 & 30.26100 & 63.74249 & 39.37592 & 0.70520 & 0.49730 \\
kknn & 9147.51425 & 27.14396 & 57.17667 & 36.86581 & 0.74955 & 0.56182 \\
mlp & 9179.79310 & 27.23974 & 57.37843 & 34.70711 & 0.78122 & 0.61031 \\
mlpe & 8746.27740 & 25.95335 & 54.66874 & 34.57953 & 0.78309 & 0.61323 \\
ksvm & \textbf{8462.61487} & \textbf{25.11162} & \textbf{52.89570} & 34.67742 & \textbf{0.79830} & \textbf{0.63729} \\
randomForest & 8807.76477 & 26.13580 & 55.05306 & 34.70615 & 0.78239 & 0.61214 \\
mr & 10410.13919 & 30.89062 & 65.06873 & 38.61068 & 0.72000 & 0.51840 \\
mars & 8842.91866 & 26.24011 & 55.27279 & \textbf{33.96852} & 0.79161 & 0.62665 \\
cubist & 9012.54150 & 26.74345 & 56.33302 & 35.70576 & 0.77611 & 0.60235 \\
pcr & 10337.63121 & 30.67546 & 64.61552 & 38.59290 & 0.72023 & 0.51873 \\
plsr & 10337.63121 & 30.67546 & 64.61552 & 38.59290 & 0.72023 & 0.51873 \\
cppls & 10337.63121 & 30.67546 & 64.61552 & 38.59290 & 0.72023 & 0.51873 \\
\bottomrule
\end{tabular}}
\end{table}

\subsection{Validación}

Se uso el paquete R Boruta propuesto \cite{kursa2010feature}, el cual es un nuevo algoritmo de selección de características 
para encontrar todas las variables relevantes. El algoritmo está diseñado como un recubrimiento alrededor 
del algoritmo de clasificación random forest. Se elimina de forma iterativa las características 
que se probaron mediante una prueba estadística para ser menos relevante que las sondas al azar. 





