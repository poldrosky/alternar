%% á é í ó ú ñ

\documentclass[conference, spanish]{IEEEtran}

\ifCLASSINFOpdf
   \usepackage[pdftex]{graphicx}
  % declare the path(s) where your graphic files are
   \graphicspath{{../pdf/}{../jpeg/}}
  % and their extensions so you won't have to specify these with
  % every instance of \includegraphics
   \DeclareGraphicsExtensions{.pdf,.jpeg,.png}
\else
  % or other class option (dvipsone, dvipdf, if not using dvips). graphicx
  % will default to the driver specified in the system graphics.cfg if no
  % driver is specified.
  \usepackage[dvips]{graphicx}
  % declare the path(s) where your graphic files are
  \graphicspath{{../eps/}}
  % and their extensions so you won't have to specify these with
  % every instance of \includegraphics
  \DeclareGraphicsExtensions{.eps}
\fi


\usepackage[utf8x]{inputenc}
\usepackage[T1]{fontenc}
\usepackage{booktabs}
\usepackage{subfigure}
\usepackage[colorlinks=true,linkcolor=black,urlcolor=black,citecolor=black, urlcolor=black, filecolor=black bookmarks=false]{hyperref}

\begin{document}

\pagestyle{empty}  

\title{Análisis de regresión para el cálculo de biomasa en el departamento de Nariño (Colombia) utilizando imágenes satelitales Landsat}

\author{\IEEEauthorblockN{GIEE}
\IEEEauthorblockA{Universidad de Nariño\\
San Juan de Pasto, Colombia\\
Email: }
\and
\IEEEauthorblockN{GIEE}
\IEEEauthorblockA{Universidad de Nariño\\
San Juan de Pasto, Colombia\\
Email: }
}

\maketitle

\begin{abstract}

En este artículo se describe la metodología utilizada para la construcción de un primer mapa del potencial de biomasa en el departamento de Nariño (Colombia) a partir de imágenes satelitales Landsat de libre acceso. Se analizan las diferentes bandas de las imágenes satelitales disponibles y su relación con bases de datos previas de biomasa aplicando diferentes técnicas de regresión y obteniendo un modelo para la generación de mapas actualizados en el departamento de Nariño.

\end{abstract}


\begin{IEEEkeywords}
biomass, regression models 
\end{IEEEkeywords}

\thispagestyle{empty} 

\IEEEpeerreviewmaketitle


\section{Introducción}

\IEEEPARstart 
En las últimas décadas la investigación en fuentes alternativas de energía ha recibido particular atención y ha pasado de tener un alto interés en los círculos académicos a convertirse en un punto prioritario en la agenda de gobiernos y organizaciones a nivel mundial.  La dependencia en combustibles fósiles y acuerdos internacionales como el protocolo de Kyoto han impulsado aún más el interés alrededor del tema.  En particular, la implementación de soluciones como páneles fotovoltáicos, parques eólicos y plantas de biomasa han atraido gran atención en especial en zonas con baja o nula cobertura.

En Colombia, según estudios del Ministerio de Minas y Energía, en el departamento de Nariño hay 15 municipios con cobertura eléctrica inferior al 80\% \cite{ministerio_de_minas_y_energia_plan_2008}. Como nueva estratégia para enfrentar esta problemática se ha propuesto el estudio y análisis de fuentes alternativas de energía en la zona. Uno de los principales objetivos es la medición y estimación de potenciales energéticos para identificar las zonas más viables en la región donde efectuar pruebas piloto y estudios de factibilidad. 

Sin embargo, uno de los principales retos para la ubicación de dichas zonas es la ausencia de bases de datos actualizadas así como series de tiempo históricas que apoyen el proceso de toma de decisiones.  Igualmente, restricciones de tiempo y costos impiden el despliegue de trabajo de campo para la recolección de información.  En el caso del análisis del potencial biomásico, estas restricciones se acentuan debido a la toma manual de muestras, extensión del área de estudio, análisis de laboratorio, dificultad del terreno e, incluso, presencia de grupos armados en la zona. 

En este sentido, diversas investigaciones han demostrado la utilidad del uso de imágenes satelitales para la generación de modelos que permitan calcular la cantidad de biomasa presente en un determinado lugar. Desde hace más de 30 años, se cuenta con acceso al repositorio de imágenes satelitales Landsat \cite{landsat} de manera libre y gratuita.  Bajo el debido tratamiento, estas imágenes pueden ser usadas para calcular valores nominales de biomasa a partir de modelos de regresión y trabajo de campo.  Sin embargo, dadas las dificultades para realizar dicho trabajo de campo, este estudio propone utilizar imágenes provistas por investigaciones anteriores.   \cite{baccini2008afirst} y \cite{baccini_estimated_2012} proporcionan bases de datos de los índices de biomasa a nivel pan-tropical entre los años 2000 y 2003.  Al igual que el conjunto de imágenes Landsat, las imágenes georreferenciadas para cada uno de los países analizados son de libre acceso y se encuentran disponibles en \cite{WHRC}.

Esta investigación presenta la metodología propuesta para la generación de un modelo de predicción de biomasa, basado en modelos de regresión e imágenes satelitales de libre acceso, y su extrapolación al resto del área de estudio.

El area de estudio de esta investigación fue el departamento de Nariño, el cual esta ubicado en el extremo Suroccidental de Colombia (en la frontera con Ecuador) con una extensión aproximada de 33.268 km, una población de 1,702 millones (según
el censo de 2013) y ubicada entre coordenadas 00° 31' 08'' y 02° 41' 08'' Norte y 76° 51' 19'' y 79° 01' 34'' Oeste (figura  \ref{fig:locationNarino}).

\begin{figure}
  \centering
  \includegraphics[width = 8cm]{locationNarino.png}
  \caption{Localización area de estudio}
  \label{fig:locationNarino}
\end{figure}
\input{relateWork_v2.0.tex}

\section{Metodología}
El repositorio de imágenes satelitales Landsat es amplio y diverso.  Si bien se constituye como una gran herramienta para la comunidad científica, su uso requiere un tratamiento previo.  De igual manera, la construcción de un modelo preliminar de biomasa a partir de información secundaria exige la selección y validación de diferentes técnicas de regresión disponibles.  Esta sección resume una metodología de cinco etapas para la construcción del modelo de biomasa para el departamento de Nariño.  La figura~\ref{fig:metodology} ilustra la metodología propuesta. A continuación se explica en más detalle cada una de las etapas.

\begin{figure}
  \centering
  \includegraphics[width = 0.5\textwidth]{metodology.png}
  \caption{Metodología}
  \label{fig:metodology}
\end{figure}

\subsection{Obtención de datos}

El proceso de obtención de datos se realizó tomando imágenes satelitales proveidas por el sensor Landsat 7 ETM+. En este proceso se descargaron 1362 imágenes satelitales desde el año 1999 hasta mediados del año 2015. Para cubrir el departamento en su totalidad fue necesario descargar imágenes de cinco escenas diferentes.  La figura \ref{fig:cuts}a detalle los respectivos identificadores (Path ID y Row ID) y extensión de cada una de las escenas usadas.

Igualmente, durante esta etapa se tuvo acceso al mapa de biomasa construido por \cite{baccini2008afirst}.  Este es un mapa con resolución espacial de 1 $Km^2$ construido a partir de un modelo basado en imágenes MODIS recolectadas durante el año 2000 y 2003.

\subsection{Preprocesamiento}

En esta etapa se realizó un trabajo básico de procesamiento sobre las imágenes adquiridas.  Primero, dada la extensión del área de estudio, las escenas descargadas tenían diferentes sistemas de coordenadas (EPSG:32618 y EPSG:32617).  Por motivos de visualización se decidió unificar el sistema de coordenadas usando EPSG:3857, popular entre las herramientas de mapeo y desarrollo de aplicaciones web.  Muchas de las escenas cubrían una gran área del Océano Pacifico así como de otros departamentos de la región.  Se recorto las imágenes para contener solo los datos referentes al departamento de Nariño.  La figura \ref{fig:cuts}b ilustra el resultado final de esta etapa.

\begin{figure}
  \centering
  \subfigure[Imágenes Satélitales de Nariño]{\label{Imágenes Satélitales Nariño} \includegraphics[width= 7cm]{cut1.png}}
  \vfill
  \subfigure[Imágenes recortadas de Nariño]{\label{Imágenes recortadas de Nariño}\includegraphics[width= 7cm]{cut2.png}}
  \caption{Prepocesamiento}
  \label{fig:cuts}
\end{figure}

De igual manera este proceso se lo realizó para el mapa de biomasa, como se muestra en la figura~\ref{fig:mapaNarino}

\begin{figure}
  \centering
  \includegraphics[width = 8cm]{mapaNarino.pdf}
  \caption{Mapa de biomasa en Nariño de 2000-2003 \cite{baccini2008afirst}}
  \label{fig:mapaNarino}
\end{figure}

\subsection{Procesamiento y limpieza de datos}

Con el objetivo de organizar los datos adquiridos se diseñó una base de datos a partir de cuatro entidades fundamentales: la fecha de la toma, los valores de reflectancia solar, los valores correspondientes de biomasa y las ubicaciones descartadas durante la limpieza de datos.  La figura \ref{fig:landsatET} ilustra el diseñó de la base de datos y los detalles de cada tabla se comentan a continuación:

\begin{figure}
  \centering
  \includegraphics[width = 8cm]{landsatET.pdf}
  \caption{Modelo entidad-relacion Landsat}
  \label{fig:landsatET}
\end{figure}

\begin{itemize}
 \item Tabla date\_landsat: en la cual se almacenan las fechas de las imágenes satelitales.
 \item Tabla reflectance: en la cual se almacenan los datos capturados y convertidos a reflectancia solar, de las bandas Landsat (1-5 y 7) y la temperatura en grados kelvin de la banda 6.
 \item Tabla discarded: en la cual se almacenan los datos que fueron descartados por diversas razones (son puntos nublados, datos ambiguos o no corresponden a vegetación).
 \item Tabla biomass: en la cual se almacenan los datos de biomasa extraídos de \cite{baccini2008afirst}.
\end{itemize}

El procesamiento de las imágenes y alimentación de la base de datos se realizó a través de scripts y archivos procesados por lotes.  Entre los procesos realizados se transformó los valores originales extraídos de las imágenes (o digital numbers) a su correspondiente valor de reflectancia solar.  Se utilizó el algoritmo propuesto en \cite{irish2000landsat} para detectar puntos nublados en la zona clasificándolos como nubes frías, calientes o ambiguas.  Estos puntos se almacenaron con la intención de realizar posteriores estudios de nubosidad.  Finalmente, se aplicó un filtro adicional para calcular en índice NDVI (Normalized Difference Vegetation Index) con el objetivo de trabajar solo con aquellos puntos relacionados con vegetación y excluir áreas como cuerpos de agua o ciudades.  La tabla~\ref{tab:datos} muestra la relación de los datos obtenidos en este proceso.

\begin{table}
\caption{Datos obtenidos en en el proceso de procesamiento y limpieza  de datos}
\label{tab:datos}
\centering
%\scalebox{0.7}{
\begin{tabular}{c c c}
\toprule
 Nombre & Valor& Detalle  \\
\midrule
Datos biomasa & 81.993 & Registros biomasa \\
 & & (años 2000 a 2003 \cite{baccini2008afirst})\\
Datos biomasa usados & 140018 & construcción del modelo\\
\hline
Imágenes Landsat& 1321 & Imágenes de Nariño\\
procesadas & & (2000 a 2014) \\
Nube caliente & 3.731.768 & Registros 2000 a 2014 \\
Nube Fria & 27.827.009 & Registros 2000 a 2014 \\
No vegetación & 3.459.210 & Registros 2000 a 2014 \\
Ambiguo & 11.987.340 & Registros 2000 a 2014 \\
\hline
Total Datos Descartados & 47.005.327 & Total datos descartados \\
Datos Validos Reflectance & 4.071.185 & Registros 2000 a 2014 \\
\hline
Datos Totales & 51.076.512 & Registros Totales\\
& & (año 2000 a 2014) \\
\bottomrule
\end{tabular}
%}
\end{table}

\subsection{Análisis de regresión}

El análisis de regresión se realizó tomando los valores de las bandas Landsat obtenidas entre los años 2000 y 2003 y adicionando los valores correspondientes de biomasa para cada ubicación, dichos valores  se extrajeron de \cite{baccini2008afirst}.  Para poder obtener una mayor confiabilidad en el modelo solo se tuvo en cuenta aquellas ubicaciones con un número significativo de muestras.  Dada la alta nubosidad de la zona, muchos de los puntos contaban con pocas lecturas.  Por lo tanto, se consolidó un nuevo conjunto de datos con el promedio de aquellos puntos que superaban al menos un número considerable de lecturas validas.  

Se construyeron diferentes modelos, iterando el número de muestras por cada punto entre 10 y 45.  El mejor modelo se obtuvo cuando el número de muestras superaba las 35 lecturas.  El conjunto de datos final arrojó 1009 registros válidos.  El comportamiento en las demás iteraciones indica que con pocas muestras el conjunto de entrada es altamente hetereogeneo, guiado por aquellos puntos con pocas lecturas y alta variabilidad.  En cambio, al aumentar el número de lecturas, la variabilidad de las muestras baja pero con el alto riesgo de sobrecargar el modelo (\cite{babyak_what_2004}). 

Antes de aplicar las técnicas de regresión, se procedió a evaluar la calidad y relevancia de las bandas de las imágenes Landsat con el objetivo de predecir valores de biomasa.  Se utilizó el algoritmo propuesto en \cite{kursa2010feature} para la extracción y evaluación de atributos. El algoritmo está diseñado como un recubrimiento alrededor del algoritmo de clasificación \texttt{random forest} y califica cada atributo en el conjunto de datos de acuerdo a su importancia a la hora de clasificar el atributo buscado. En la figura~\ref{fig:boruta} se puede observar la relevancia de todas las bandas Landsat (en verde) que se ubican por encima de los valores por defecto (en azul).  Esto indica que la relevancia de las bandas de las imágenes Landsat esta por encima del azar. Concluimos que todas las bandas resultan importantes a la hora de modelar biomasa.

\begin{figure}
  \centering
  \includegraphics[width = 8cm]{boruta.pdf}
  \caption{Relevancia de bandas Landsat en el análisis de regresión}
  \label{fig:boruta}
\end{figure}

Con un nuevo conjunto de datos definido y evaluado, se continuó con la construcción de modelos de regresión utilizando diversas técnicas de análisis.  La biblioteca de código abierto \texttt{rminer}, presentada por \cite{cortez2010data} para la herramienta R, provee diferentes implementaciones y una interfaz que facilita la ejecución de diferentes pruebas y la extracción de modelos y sus correspondientes métricas de evaluación.  

Se construyó un total de 13 modelos y se evaluaron 6 metricas por cada uno. La tabla \ref{tab:metricas} ilustra los resultados obtenidos. Las técnicas utilizadas fueron: ctree (conditional inference tree), rpart (decision tree), kknn (k-nearest neighbor), mlp (multilayer perceptron with one hidden layer), mlpe (multilayer perceptron ensemble), ksvm (support vector machine), randomForest (random forest algorithm), mr (multiple regression), mars (multivariate adaptive regression splines), cubist (rule-based model), pcr (principal component regression ), plsr (partial least squares regression) y cppls (canonical powered partial least squares).  Las metricas evaluadas fueron: SAE (sum absolute error), MAE (mean absolute error), RAE (relative absolute error), RMSE (root mean squared error), COR (correlation) y R2 (coefficient of determination $R^2$).

\begin{table}
\caption{Métricas de modelos analizados. Los valores en negrilla indican los mejores resultados para cada métrica.}
\label{tab:metricas}
\centering
\scalebox{0.7}{
\begin{tabular}{c c c c c c c}
\toprule
 & SAE& MAE & RAE & RMSE & COR & R2 \\
\midrule
ctree & 10406.58225 & 30.88007 & 65.04650 & 40.02893 & 0.69401 & 0.48165 \\
rpart & 10197.95826 & 30.26100 & 63.74249 & 39.37592 & 0.70520 & 0.49730 \\
kknn & 9147.51425 & 27.14396 & 57.17667 & 36.86581 & 0.74955 & 0.56182 \\
mlp & 9179.79310 & 27.23974 & 57.37843 & 34.70711 & 0.78122 & 0.61031 \\
mlpe & 8746.27740 & 25.95335 & 54.66874 & 34.57953 & 0.78309 & 0.61323 \\
ksvm & \textbf{8462.61487} & \textbf{25.11162} & \textbf{52.89570} & 34.67742 & \textbf{0.79830} & \textbf{0.63729} \\
randomForest & 8807.76477 & 26.13580 & 55.05306 & 34.70615 & 0.78239 & 0.61214 \\
mr & 10410.13919 & 30.89062 & 65.06873 & 38.61068 & 0.72000 & 0.51840 \\
mars & 8842.91866 & 26.24011 & 55.27279 & \textbf{33.96852} & 0.79161 & 0.62665 \\
cubist & 9012.54150 & 26.74345 & 56.33302 & 35.70576 & 0.77611 & 0.60235 \\
pcr & 10337.63121 & 30.67546 & 64.61552 & 38.59290 & 0.72023 & 0.51873 \\
plsr & 10337.63121 & 30.67546 & 64.61552 & 38.59290 & 0.72023 & 0.51873 \\
cppls & 10337.63121 & 30.67546 & 64.61552 & 38.59290 & 0.72023 & 0.51873 \\
\bottomrule
\end{tabular}}
\end{table}

\subsection{Construcción de mapas}

A partir de los resultados de la tabla \ref{tab:metricas} durante la construcción de mapas se utilizó el modelo \texttt{ksvm}.  Para la construcción de mapas de biomasa se utilizó el método Kriging (\cite{bivand_applied_2013, cressie_statistics_2015}).  Kriging provee una solución al problema de la estimación basada en un modelo continuo de variación espacial estocástica.  El objetivo de Kriging es el de estimar el valor de una variable aleatoria, en este caso biomasa, en uno o más puntos no muestreados o sobre grandes bloques. 

El método Kriging recibe como entrada un muestreo de datos y una malla dependiendo de la resolución que se quiera obtener. Para ello, se obtuvo una muestra de los datos obtenidos al aplicar el modelo seleccionado a datos agregados por mes y año y un mapa general que abarca el periodo de estudio.  La malla se construyó con puntos regulares espaciados cada 450 metros. En la figura~\ref{fig:biomasaMes}, figura~\ref{fig:biomasaAnio}, figura~\ref{fig:biomasaTotal}  se muestra los mapas obtenidos por meses, años y general respectivamente. 
 
\begin{figure}
  \centering
  \includegraphics[width = 0.45\textwidth]{mapMonthsBiomass.pdf}
  \caption{Mapas de biomasa por mes}
  \label{fig:biomasaMes}
\end{figure}

\begin{figure}
  \centering
  \includegraphics[width = 0.45\textwidth]{mapYearsBiomass.pdf}
  \caption{Mapas de biomasa por año}
  \label{fig:biomasaAnio}
\end{figure}

\begin{figure}
  \centering
  \includegraphics[width = 0.45\textwidth]{mapGeneralBiomass.pdf}
  \caption{Mapa de biomasa general entre los años 2000 y 2014}
  \label{fig:biomasaTotal}
\end{figure}

%\section{Conclusiones}

Se construyeron mapas energéticos con el componente solar en el departamento de Nariño,
usando los sensores Landsat 7 y MODIS.

Las imágenes sateliatles son una gran fuente de información debido a la capacidad de almacenar gran cantidad de registros históricos para diferentes tipos de datos, estos datos
poco a poco estan siendo utilizados por organizaciones para determinar características terrestres, fenómenos naturales, condiciones de los mares,
características de la vegetación, etc. Por esta razón el uso de imágenes satelitales
en la investigación da resultados aproximados y a bajo costo, teniendo en cuenta el costo
que puede implicar hacer muestreo en campo.

Se construyo una metodología para la construcción de mapas energéticos con el potencial
de radiación solar, el cual se lo puede aplicar en zonas donde no tengan estaciones climáticas con los sensores Landsat o MODIS.

En la construcción del mejor modelo, con ambos sensores Landsat 7 y MODIS se tubo un $R^2$ por encima del 90\%,
teniendo en cuenta que las imágenes landsat 7 se obtienen cada 16 dias y las de MODIS son diarias, sería
de mayor provecho usar el sensor MODIS para hacer estudios posteriores con series de tiempo.

Tanto la correlación como el $R^2$ entre los mapas construidos a partir del sensonr Landsat 7 y MODIS
es buena, teniendo en cuenta que en el mapa general se tiene una correlación del 94\% y un $R^2$ del
88\%.



%\appendices
%\section{Repositorio}
%El código fuente y conjunto de datos se encuentran en el repositorio de github.


\ifCLASSOPTIONcompsoc
  % The Computer Society usually uses the plural form
  \section*{Agradecimientos}
\else
  % regular IEEE prefers the singular form
  \section*{Agradecimientos}
\fi

Esta investigación se hizo posible gracias a los recursos otorgados por el Sistemas General de Regalias en el marco de proyecto ``Análisis de Oportunidades Energéticas con Fuentes Alternativas en el Departamento de Nariño'' ejecutado por el programa de Ingeniería Electrónica de la Universidad de Nariño.

% Can use something like this to put references on a page
% by themselves when using endfloat and the captionsoff option.
\ifCLASSOPTIONcaptionsoff
  \newpage
\fi


\bibliographystyle{plain}
\bibliography{bibliography}

\end{document}
