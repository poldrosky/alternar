
\section{Metodología}
El repositorio de imágenes satelitales Landsat es amplio y diverso.  Si bien se constituye como una gran herramienta para la comunidad científica, su uso requiere un tratamiento previo.  De igual manera, la construcción de un modelo preliminar de biomasa a partir de información secundaria exigue la selección y validación de diferentes técnicas de regresión disponibles.  Esta sección resume una metodología de cinco etapas para la construcción del modelo de biomasa para el departamento de Nariño.  La figura~\ref{fig:metodology} ilustra la metodología propuesta. A continuación se explica en más detalle cada una de las etapas.

\begin{figure}
  \centering
  \includegraphics[width = 0.5\textwidth]{metodology.png}
  \caption{Metodología}
  \label{fig:metodology}
\end{figure}

\subsection{Obtención de datos}

El proceso de obtención de datos se realizó tomando imágenes satelitales proveidas por el sensor Landsat 7 ETM+. En este proceso se descargaron 1362 imágenes satelitales desde el año 1999 hasta mediados del año 2015. Para cubrir el departamento en su totalidad fue necesario descargar imágenes de cinco escenas diferentes.  La figura \ref{fig:cuts}a detalle los respectivos identificadores (Path ID y Row ID) y extensión de cada una de las escenas usadas.

Igualmente, durante esta etapa se tuvo acceso al mapa de biomasa construido por \cite{baccini2008afirst}.  Este es un mapa con resolución espacial de 1 $Km^2$ construido a partir de un modelo basado en imágenes MODIS recolectadas durante el año 2000 y 2003.

\subsection{Preprocesamiento}

En esta etapa se realizó un trabajo básico de procesamiento sobre las imágenes adquiridas.  Primero, dada la extensión del área de estudio, las escenas descargadas tenían diferentes sistemas de coordenadas (EPSG:32618 y EPSG:32617).  Por motivos de visualización se decidió unificar el sistema de coordenadas usando EPSG:3857, popular entre las herramientas de mapeo y desarrollo de aplicaciones web.  Muchas de las escenas cubrían una gran área del Océano Pacifico así como de otros departamentos de la región.  Se recorto las imágenes para contener solo los datos referentes al departamento de Nariño.  La figura \ref{fig:cuts}b ilustra el resultado final de esta etapa.

\begin{figure}
  \centering
  \subfigure[Imágenes Satélitales de Nariño]{\label{Imágenes Satélitales Nariño} \includegraphics[width= 7cm]{cut1.png}}
  \vfill
  \subfigure[Imágenes recortadas de Nariño]{\label{Imágenes recortadas de Nariño}\includegraphics[width= 7cm]{cut2.png}}
  \caption{Prepocesamiento}
  \label{fig:cuts}
\end{figure}

De igual manera este proceso se lo realizó para el mapa de biomasa, como se muestra en la figura~\ref{fig:mapaNarino}

\begin{figure}
  \centering
  \includegraphics[width = 8cm]{mapaNarino.pdf}
  \caption{Mapa de biomasa en Nariño de 2000-2003 \cite{baccini2008afirst}}
  \label{fig:mapaNarino}
\end{figure}

\subsection{Procesamiento y limpieza de datos}

Se diseñó una base de datos para capturar los datos,
como lo muestra la figura~\ref{fig:landsatET}, la cual tiene 4 tablas. 

\begin{figure}
  \centering
  \includegraphics[width = 8cm]{landsatET.pdf}
  \caption{Modelo entidad-relacion Landsat}
  \label{fig:landsatET}
\end{figure}

Tabla date\_landsat: en la cual se almacenan las fechas de las imágenes satelitales.

Tabla reflectance: en la cual se almacenan los datos capturados y convertidos en reflentance,
de las bandas landsat (1 - 5,7) y la temperatura en grados kelvin de la banda 6.

Tabla discarded: en la cual se almacenan datos que fueron descartados, por varias razones,
son nubes calientes, nubes frias, datos ambiguos o no son vegetación.

Tabla biomass: en la cual se almacenan los datos de biomassa del mapa de \cite{baccini2008afirst}.

Para procesar las imágenes y llenar la base de datos se realizó un Script, el cual captura el Digital Number
de las imágenes satélitales y lo transforma en valor en reflectance. En este procesamiento de imagenes, se adiciono al Script unos filtros para para detección de nubes calientes,
nubes, frias, datos ambiguos como lo muestra el algoritmo propuesto por \cite{irish2000landsat}, además se aplico 
un filtro adicional, el NVDI(normalized difference vegetation index) para trabajar unicamente con datos de vegetación.

La tabla~\ref{tab:datos} muestra la relación de los datos obtenidos en este proceso.

\begin{table}
\caption{Datos obtenidos en en el proceso de procesamiento y limpieza  de datos}
\label{tab:datos}
\centering
\scalebox{0.7}{
\begin{tabular}{c c c}
\toprule
 Nombre & Valor& Detalle  \\
\midrule
Datos biomasa & 81.993 & Registros de biomasa para año 2000 a 2003 de  \cite{baccini2008afirst}\\
Datos biomasa usados & 140018 & Registros para construir el modelo grilla 450 metros\\
\hline
Imágenes landsat procesadas & 1321 & Imágenes de Nariño de 2000 a 2014 \\
Nube caliente & 3.731.768 & Registros de 2000 a 2014 \\
Nube Fria & 27.827.009 & Registros de 2000 a 2014 \\
No vegetacion & 3.459.210 & Registros de 2000 a 2014 \\
Ambiguo & 11.987.340 & Registros de 2000 a 2014 \\
\hline
Total Datos Descartados & 47.005.327 & Total datos descartados \\
Datos Validos Reflectance & 4.071.185 & Registros de 2000 a 2014 \\
\hline
Datos Totales & 51.076.512 & Registros Totales desde año 2000 a 2014 \\
\bottomrule
\end{tabular}}
\end{table}

\subsection{Análisis de regresión}

El análisis de regresión se realizó tomando los valores de las bandas landsat obtenidas año 2000 y 2003 y el valor de biomasa obtenido en \cite{baccini2008afirst},
para poder obtener un mejor modelo se agrupó y se saco un promedio con valores de las bandas landsat en cada punto, se fue iterando con  valores que superaban al menos N número de
muestras, siendo N desde 1 hasta 45 muestras, el mejor modelo obtenido fué cuando el número de muestas en cada punto superaba al menos las 35 muestras, este conjunto
de datos obtenido tenía 1009 registros. El comportamiento en las demás iteraciones muestra que con menos muestras hay más registros y eso hace que no se encuentre un buen modelo,
pero cuando hay mas muestras los registros son menores y esto también hace que el resultado del modelo tampoco sea bueno.

En la tabla~\ref{tab:metricas} se muestra las métricas de los modelos analizadas con 35 muestras y 1009 datos, el cual es el mejor modelo, esta tabla 
se la realizó usando la biblioteca de código abierto rminer presentada por \cite{cortez2010data} para la herramienta R.

\begin{table}
\caption{Métricas de modelos analizados con 35 muestras y 1009 datos}
\label{tab:metricas}
\centering
\scalebox{0.7}{
\begin{tabular}{c c c c c c c}
\toprule
 & SAE& MAE & RAE & RMSE & COR & R2 \\
\midrule
ctree & 10406.58225 & 30.88007 & 65.04650 & 40.02893 & 0.69401 & 0.48165 \\
rpart & 10197.95826 & 30.26100 & 63.74249 & 39.37592 & 0.70520 & 0.49730 \\
kknn & 9147.51425 & 27.14396 & 57.17667 & 36.86581 & 0.74955 & 0.56182 \\
mlp & 9179.79310 & 27.23974 & 57.37843 & 34.70711 & 0.78122 & 0.61031 \\
mlpe & 8746.27740 & 25.95335 & 54.66874 & 34.57953 & 0.78309 & 0.61323 \\
ksvm & \textbf{8462.61487} & \textbf{25.11162} & \textbf{52.89570} & 34.67742 & \textbf{0.79830} & \textbf{0.63729} \\
randomForest & 8807.76477 & 26.13580 & 55.05306 & 34.70615 & 0.78239 & 0.61214 \\
mr & 10410.13919 & 30.89062 & 65.06873 & 38.61068 & 0.72000 & 0.51840 \\
mars & 8842.91866 & 26.24011 & 55.27279 & \textbf{33.96852} & 0.79161 & 0.62665 \\
cubist & 9012.54150 & 26.74345 & 56.33302 & 35.70576 & 0.77611 & 0.60235 \\
pcr & 10337.63121 & 30.67546 & 64.61552 & 38.59290 & 0.72023 & 0.51873 \\
plsr & 10337.63121 & 30.67546 & 64.61552 & 38.59290 & 0.72023 & 0.51873 \\
cppls & 10337.63121 & 30.67546 & 64.61552 & 38.59290 & 0.72023 & 0.51873 \\
\bottomrule
\end{tabular}}
\end{table}

En este proceso tambien se usó el paquete R Boruta \cite{kursa2010feature}, el cual es un nuevo algoritmo de selección de características 
para encontrar todas las variables relevantes. El algoritmo está diseñado como un recubrimiento alrededor 
del algoritmo de clasificación random forest. Esto para saber si todas las bandas de landsat utilizadas eran relevantes para encontrar biomasa, 
en la figura~\ref{fig:boruta} se puede observar la relevancia de las bandas landsat para encontrar biomasa en el mejor modelo.

\begin{figure}
  \centering
  \includegraphics[width = 8cm]{boruta.pdf}
  \caption{Relevancia de bandas landsat en el análisis de regresión}
  \label{fig:boruta}
\end{figure}

\subsection{Construcción de mapas}

Para la construcción de mapas de biomasa se utilizó el método Kriging que provee una solución al problema 
de la estimación basada en un modelo continuo de variación espacial estocástica, el objetivo de Kriging es el de estimar el valor de una 
variable aleatoria, Z, en uno o más puntos no muestreados o sobre grandes bloques. 

El método Kriging recibe como entrada datos de la muestra, y una malla dependiendo de la resolución que se quiera obtener, por ello los datos de muestra se
obtuvieron aplicando el modelo obtenido en el análisis de regresión a datos agrupados en cada punto por mes, año y uno general entre el año 2000 a 2014; y la 
 malla se construyó con puntos regulares espaciados cada 450 metros. 
 
 En la figura~\ref{fig:biomasaMes}, figura~\ref{fig:biomasaAnio}, figura~\ref{fig:biomasaTotal}  se muestra los mapas obtenidos por meses, años y general entre el año 2000 a 2014 respectivamente. 
 
\begin{figure}
  \centering
  \includegraphics[width = 8cm]{mapMonthsBiomass.pdf}
  \caption{Mapas biomasa por meses}
  \label{fig:biomasaMes}
\end{figure}

\begin{figure}
  \centering
  \includegraphics[width = 8cm]{mapYearsBiomass.pdf}
  \caption{Mapas biomasa por años}
  \label{fig:biomasaAnio}
\end{figure}

\begin{figure}
  \centering
  \includegraphics[width = 8cm]{mapGeneralBiomass.pdf}
  \caption{Mapas biomasa general años 2000-2014}
  \label{fig:biomasaTotal}
\end{figure}
