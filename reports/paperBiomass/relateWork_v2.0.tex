\section{Trabajos Relacionados}

El estudio de índices de biomasa ha sido ampliamente registrado por diferentes estudios. Muchos de ellos demuestran la utilidad del uso de imágenes satelitales a diferentes resoluciones. Por lo general, estos estudios parten de un trabajo de campo donde se calcula el valor nominal de biomasa a diferentes muestras tomadas de manera manual y haciendo uso de técnicas tradicionales de laboratorio. Posteriormente, se utilizan estos resultados y las diferentes bandas proveídas por las imágenes de satélite para inferir un modelo utilizando alguna técnica de regresión que después es extrapolada al resto del área de estudio.

Por ejemplo, \cite{klemas2013remotesensing} usa esta metodología para detectar cambios en los niveles de biomasa en diferentes zonas costeras de los Estados Unidos utilizando imágenes LIDAR\footnote{Una tecnica de muestreo que mide distancia haciendo uso de rayos laser.} y regresión lineal. De forma análoga, \cite{baccini2008afirst} usa imágenes MODIS\footnote{Moderate Resolution Imaging Spectroradiometer  - \url{http://modis.gsfc.nasa.gov/}} y árboles de decisión (en adición a las técnicas tradicionales de regresión) para estimar el índice AGB (Above-Ground Biomass) en una extensa área del África tropical. Similiar a este trabajo, \cite{mitchard2009usingsatellite} utilizan imágenes de radar para predecir AGB en cuatro reservas y parques nacionales africanos clasificando diferentes tipos de corteza terrestre.

\cite{muukkonen2007biomass} hacen también uso de imágenes MODIS en conjunto con imágenes ASTER\footnote{Advanced Spaceborne Thermal Emission and Reflection Radiometer - \url{http://asterweb.jpl.nasa.gov/}} para estimar biomasa con el fin de levantar un inventario de captura de carbono. Un aporte importante de esta publicación es que comparten la metodología utilizada durante el proyecto. \cite{powell2010quantification} introducen el uso de nuevas técnicas de regresión (reduced major axis regression, gradient nearest neighbor imputation y ramdom forest regression trees) para la generación de modelos de biomasa esta vez analizando imágenes Landsat.

En \cite{hall2006modeling} se introduce bioSTRUCT, un método para generar correlaciones entre los valores continuos medidos por las bandas de las imágenes satelitales y el AGB medido previamente usando técnicas de laboratorio. El artículo ilustra la metodología con un caso de estudio en Alberta (Canada) e imágenes Landsat ETM+ de libre acceso. Como resultado se obtienen formulas de regresión a partir de un número limitado de muestras que pueden extrapolarse al resto del área de estudio.


