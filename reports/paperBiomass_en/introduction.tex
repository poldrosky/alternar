\section{Introduction} 

\IEEEPARstart According with studies of Ministry of Mines and Energy of Colombia, in the department of Nariño 15 municipalities have electric coverage under 80\% \cite{ministerio_de_minas_y_energia_plan_2008}. As new strategy to address this situation was defined the mensuration and estimation of energy potential at more viable region zones. One component to analyze is biomass potential to electricity generation. However, one problem stated to locate propitious places is the lack of updated databases in this study area which allow their analysis.

Several research have shown how advantageous is use satellite imagery to generate models which allow to calculate how much biomass exists in an specific place. For more than 30 years, the access to Landsat satellite imagery repository is granted \cite{landsat}; this images, with appropriated treatment, can be used to calculate nominal values of biomass. however, this models require fieldwork in selected zones to infer initial formulas from a sample measurement. Due the difficulties  to execute this fieldwork were used images provided by previous researches \cite{baccini2008afirst}, \cite{baccini_estimated_2012} where are provided biomass levels at pantropical level. The access to images to each country used for this research are available at \cite{WHRC}.

This research is oriented to fulfill the requirements to generate a model for biomass prediction and its extrapolation to the full study area.