\documentclass[conference]{IEEEtran}

\ifCLASSINFOpdf
   \usepackage[pdftex]{graphicx}
  % declare the path(s) where your graphic files are
   \graphicspath{{../pdf/}{../jpeg/}}
  % and their extensions so you won't have to specify these with
  % every instance of \includegraphics
   \DeclareGraphicsExtensions{.pdf,.jpeg,.png}
\else
  % or other class option (dvipsone, dvipdf, if not using dvips). graphicx
  % will default to the driver specified in the system graphics.cfg if no
  % driver is specified.
  \usepackage[dvips]{graphicx}
  % declare the path(s) where your graphic files are
  \graphicspath{{../eps/}}
  % and their extensions so you won't have to specify these with
  % every instance of \includegraphics
  \DeclareGraphicsExtensions{.eps}
\fi


\usepackage[utf8x]{inputenc}
\usepackage[T1]{fontenc}
\usepackage{lmodern}
\usepackage{booktabs}
\usepackage{multirow}
\usepackage{algorithmic}
\usepackage[Algoritmo]{algorithm}
\usepackage[spanish,USenglish]{babel}
\usepackage[colorlinks=true,linkcolor=black,urlcolor=black,citecolor=black, urlcolor=black, filecolor=black bookmarks=false]{hyperref}
\usepackage{subfigure}


\addto\captionsspanish{
 \def\tablename{Tabla}}

\begin{document}

\pagestyle{empty}  

\selectlanguage{spanish}

\title{Regression analysis to calculate biomass of the department of Nariño (Colombia) using Landsat satellite imagery }

\author{\IEEEauthorblockN{GIEE}
\IEEEauthorblockA{University of Nariño\\
San Juan de Pasto, Colombia\\
Email: }
\and
\IEEEauthorblockN{GIIWW}
\IEEEauthorblockA{University of Nariño\\
San Juan de Pasto, Colombia\\
Email: }
}

\maketitle

\selectlanguage{USenglish}
\begin{abstract}

This paper describes the basic components of the applied research which aims to build the first biomass potential map of the department of Nariño (Colombia) using Landsat open source satellite imagery. The different band are analyzed from available satellite imagery and their relation with previous biomass data bases applying different regression techniques; and generating a model to creation of updated maps at the department of Nariño.

\end{abstract}
 
\selectlanguage{spanish}


\begin{IEEEkeywords}
biomass, regression models 
\end{IEEEkeywords}

\thispagestyle{empty} 

\IEEEpeerreviewmaketitle


\section{Introducción}

\IEEEPARstart{E}{l} proyecto general ``Análisis de oportunidades energéticas con fuentes
alternativas en el departamento de Nariño'' se ha planteado como uno de sus
objetivos la medición y estimación de potenciales energéticos en zonas viables
de la región. Uno de los componentes a analizar es el potencial de biomasa para
la generación eléctrica. Uno de los problemas que se plantea para la ubicación
de lugares propicios es la ausencia de bases de datos históricas en el área de
estudio que permitan su respectivo análisis.
Diversas investigaciones han demostrado la utilidad del uso de imágenes
satelitales para la generación de modelos que permitan calcular la cantidad de
biomasa presente en un determinado lugar. En la actualidad, se cuenta con
libre acceso al repositorio de imágenes satelitales Landsat que con el debido
tratamiento pueden ser usadas para calcular valores nominales de biomasa. Sin
embargo, dichos modelos requieren la ejecución de trabajo de campo en la zona
para inferir fórmulas iniciales a partir de la medición tradicional de una muestra.
Esta investigación esta orientada a cumplir con los requerimientos necesarios
para la generación de un modelo de predicción de biomasa y su extrapolación
al resto del área de estudio.

\section{Trabajos relacionados}

Diferentes estudios han explorado la construcción de mapas eólicos a partir
 de muestras tomadas en terreno.
 Por ejemplo, \cite{haslett1989spacetime} utilizan técnicas de auto-correlación espacio-temporal, para estimar la
 potencia generada por turbinas en Irlanda a partir de pocos datos de entrada.
 Adicionales técnicas de interpolación espacial fueron utilizadas por \cite{luo2008acomparison} 
 para generar mapas de viento de alta resolución en el Reino Unido.
 El objetivo perseguido por la investigación era comparar y evaluar diversos
 métodos de interpolación para seleccionar el más adecuado.
 En los Países Bajos, \cite{stepek2011interpolating}  utilizan un modelo llamado de bicapa para estimar la velocidad del viento
 a partir de 31 estaciones meteorológicas.

Sin embargo, es importante resaltar que las características de velocidad
 y dirección del viento no son lo únicos criterios a tener en cuenta a la
 hora de escoger las mejore ubicaciones.
 La evaluación multi-criterio (MCE) ha tenido una amplia acogida a la hora
 de evaluar características físicas junto con otros atributos como aspectos
 económicos y sociales.
 
\cite{rodman2006ageographic}  utilizan técnicas MCE para determinar posibles ubicaciones de turbinas
 en el norte de California evaluando componentes físicos,ambientales y humanos. \cite{janke2010multicriteria}
 categoriza diferentes aspectos de acuerdo al potencial eólico y solar e
 identifica áreas susceptibles a la instalación de turbinas y paneles solares
 en Colorado.
 
 Entre los aspectos evaluados se encuentran la distancia a carreteras y
 líneas de transmisión eléctrica, coberturas del terreno, densidad de población
 y áreas protegidas por la ley. 
 
 \cite{petrov2014utilization} explora nuevos algoritmos de análisis de datos para modelar la ubicación
 de turbinas en Iowa apoyándose en un sistema espacial multi-criterio de
 soporte a la toma de decisiones.
 Nuevas técnicas de análisis de datos, comúnmente conocidas como minería
 de datos, han demostrado muy buenos resultados a la hora de modelar fenómenos
 atmosféricos.
 
 Por ejemplo, \cite{yusof2014miningfrequent} utiliza algoritmos para detectar patrones secuenciales en series de tiempo
 de viento desde estaciones en los Países Bajos para detectar anomalías
 en el flujo, velocidad y dirección del viento.


\section{Medotología}


Antes de aplicar técnicas de detección de patrones secuenciales se hace necesario 
seleccionar y limpiar las series de tiempo disponibles para asegurar que
se ajusten al formato necesario. De igual manera, se requiere que las series de
tiempo se agrupen dentro un conjunto de datos unificado al cual se aplicaran las
primeras técnicas de exploración de datos para entender calidad de los datos,
las posibilidades de análisis y plantear una primera estrategia de análisis.
Al mismo tiempo, es importante identificar y estudiar las técnicas y algoritmos
 de detección de patrones secuenciales presentes en la literatura. Conocer
al detalle las diferentes implementaciones, casos de estudio y fundamentación
teórica de los mismos resulta vital durante la fase de ejecución. Una vez los
algoritmos hayan sido seleccionados y documentados se aplicarán a las series de
tiempo previamente depuradas.
Los resultados de esta ejecución exigirán una etapa de interpretación y discusión
 de los resultados donde se deberán seleccionar o implementar diferentes
3técnicas de visualización. Frecuentemente, al aplicar técnicas de minería de datos
se produce una gran cantidad de resultados donde las técnicas de visualización y
filtrado resultan fundamentales para el la compresión del conocimiento generado.
La interpretación de los resultados se plasmará en un reporte de investigación
como base para la publicación de un artículo científico como resultado de la
investigación.

\subsection{Recopilación de datos}

La primera parte de la investigación consistió en la identificación y exploración
 de las fuentes de datos disponibles. Se utilizó las
bases de datos eólicas de un proveedor externo (Vaisala Inc) y se trabajó
en una estrategia de muestreo para la obtención de una muestra significativa de
series de tiempo con la cual se espera construir los mapas de potencial eólico
necesarios a 50 y 120 metros de altura. 

Para este muestreo se obtuvieron 480 series de tiempo a 50 y 120 metros, regularmente
espaciados en el departamento de Nariño del proveedor Vaisala Inc, conjunto de reánalisis de datos MERRA(Modern Era Retrospective-analysis for Research and Applications) la cual
tiene una mayor resolucion espacial y temporal (por hora), cada punto contiene
atributos como: latitude, longitud, timestamp, presión, temperatura, velocidad y direccion del viento



\subsection{Limpieza de datos}

\subsection{Selección de técnicas de análisis}

\subsection{Detección de patrones}

\subsection{Visualización y análisis de resultados}



\section{Conclusiones}

Se construyeron mapas energéticos con el componente solar en el departamento de Nariño,
usando los sensores Landsat 7 y MODIS.

Las imágenes sateliatles son una gran fuente de información debido a la capacidad de almacenar gran cantidad de registros históricos para diferentes tipos de datos, estos datos
poco a poco estan siendo utilizados por organizaciones para determinar características terrestres, fenómenos naturales, condiciones de los mares,
características de la vegetación, etc. Por esta razón el uso de imágenes satelitales
en la investigación da resultados aproximados y a bajo costo, teniendo en cuenta el costo
que puede implicar hacer muestreo en campo.

Se construyo una metodología para la construcción de mapas energéticos con el potencial
de radiación solar, el cual se lo puede aplicar en zonas donde no tengan estaciones climáticas con los sensores Landsat o MODIS.

En la construcción del mejor modelo, con ambos sensores Landsat 7 y MODIS se tubo un $R^2$ por encima del 90\%,
teniendo en cuenta que las imágenes landsat 7 se obtienen cada 16 dias y las de MODIS son diarias, sería
de mayor provecho usar el sensor MODIS para hacer estudios posteriores con series de tiempo.

Tanto la correlación como el $R^2$ entre los mapas construidos a partir del sensonr Landsat 7 y MODIS
es buena, teniendo en cuenta que en el mapa general se tiene una correlación del 94\% y un $R^2$ del
88\%.



%\appendices
%\section{Repositorio}
%El código fuente y conjunto de datos se encuentran en el repositorio de github.


\ifCLASSOPTIONcompsoc
  % The Computer Society usually uses the plural form
  \section*{Acknowledgments}
\else
  % regular IEEE prefers the singular form
  \section*{Acknowledgment}
\fi

This research was possible thanks to resources given by the General Royalty System in the framework of the project ``Analysis of energy opportunities from alternative sources at the department of Nariño'' implemented by the Electronic Engineering program of the University of Nariño.

% Can use something like this to put references on a page
% by themselves when using endfloat and the captionsoff option.
\ifCLASSOPTIONcaptionsoff
  \newpage
\fi


\bibliographystyle{IEEEtran}

\bibliography{IEEEabrv,bibliography}



\end{document}