\section{Conclusiones}

Se construyeron mapas energéticos con el componente solar en el departamento de Nariño,
usando los sensores Landsat 7 y MODIS.

Las imágenes sateliatles son una gran fuente de información debido a la capacidad de almacenar gran cantidad de registros históricos para diferentes tipos de datos, estos datos
poco a poco estan siendo utilizados por organizaciones para determinar características terrestres, fenómenos naturales, condiciones de los mares,
características de la vegetación, etc. Por esta razón el uso de imágenes satelitales
en la investigación da resultados aproximados y a bajo costo, teniendo en cuenta el costo
que puede implicar hacer muestreo en campo.

Se construyo una metodología para la construcción de mapas energéticos con el potencial
de radiación solar, el cual se lo puede aplicar en zonas donde no tengan estaciones climáticas con los sensores Landsat o MODIS.

En la construcción del mejor modelo, con ambos sensores Landsat 7 y MODIS se tubo un $R^2$ por encima del 90\%,
teniendo en cuenta que las imágenes landsat 7 se obtienen cada 16 dias y las de MODIS son diarias, sería
de mayor provecho usar el sensor MODIS para hacer estudios posteriores con series de tiempo.

Tanto la correlación como el $R^2$ entre los mapas construidos a partir del sensonr Landsat 7 y MODIS
es buena, teniendo en cuenta que en el mapa general se tiene una correlación del 94\% y un $R^2$ del
88\%.
