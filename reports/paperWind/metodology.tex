\section{Medotología}


Antes de aplicar técnicas de detección de patrones secuenciales se hace necesario 
seleccionar y limpiar las series de tiempo disponibles para asegurar que
se ajusten al formato necesario. De igual manera, se requiere que las series de
tiempo se agrupen dentro un conjunto de datos unificado al cual se aplicaran las
primeras técnicas de exploración de datos para entender calidad de los datos,
las posibilidades de análisis y plantear una primera estrategia de análisis.
Al mismo tiempo, es importante identificar y estudiar las técnicas y algoritmos
 de detección de patrones secuenciales presentes en la literatura. Conocer
al detalle las diferentes implementaciones, casos de estudio y fundamentación
teórica de los mismos resulta vital durante la fase de ejecución. Una vez los
algoritmos hayan sido seleccionados y documentados se aplicarán a las series de
tiempo previamente depuradas.
Los resultados de esta ejecución exigirán una etapa de interpretación y discusión
 de los resultados donde se deberán seleccionar o implementar diferentes
3técnicas de visualización. Frecuentemente, al aplicar técnicas de minería de datos
se produce una gran cantidad de resultados donde las técnicas de visualización y
filtrado resultan fundamentales para el la compresión del conocimiento generado.
La interpretación de los resultados se plasmará en un reporte de investigación
como base para la publicación de un artículo científico como resultado de la
investigación.

\subsection{Recopilación de datos}

La primera parte de la investigación consistió en la identificación y exploración
 de las fuentes de datos disponibles. Se utilizó las
bases de datos eólicas de un proveedor externo (Vaisala Inc) y se trabajó
en una estrategia de muestreo para la obtención de una muestra significativa de
series de tiempo con la cual se espera construir los mapas de potencial eólico
necesarios a 50 y 120 metros de altura. 

Para este muestreo se obtuvieron 480 series de tiempo a 50 y 120 metros, regularmente
espaciados en el departamento de Nariño del proveedor Vaisala Inc, conjunto de reánalisis de datos MERRA(Modern Era Retrospective-analysis for Research and Applications) la cual
tiene una mayor resolucion espacial y temporal (por hora), cada punto contiene
atributos como: latitude, longitud, timestamp, presión, temperatura, velocidad y direccion del viento



\subsection{Limpieza de datos}

\subsection{Selección de técnicas de análisis}

\subsection{Detección de patrones}

\subsection{Visualización y análisis de resultados}


