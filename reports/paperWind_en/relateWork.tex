\section{Trabajos relacionados}

Diferentes estudios han explorado la construcción de mapas eólicos a partir
 de muestras tomadas en terreno.
 Por ejemplo, \cite{haslett1989spacetime} utilizan técnicas de auto-correlación espacio-temporal, para estimar la
 potencia generada por turbinas en Irlanda a partir de pocos datos de entrada.
 Adicionales técnicas de interpolación espacial fueron utilizadas por \cite{luo2008acomparison} 
 para generar mapas de viento de alta resolución en el Reino Unido.
 El objetivo perseguido por la investigación era comparar y evaluar diversos
 métodos de interpolación para seleccionar el más adecuado.
 En los Países Bajos, \cite{stepek2011interpolating}  utilizan un modelo llamado de bicapa para estimar la velocidad del viento
 a partir de 31 estaciones meteorológicas.

Sin embargo, es importante resaltar que las características de velocidad
 y dirección del viento no son lo únicos criterios a tener en cuenta a la
 hora de escoger las mejore ubicaciones.
 La evaluación multi-criterio (MCE) ha tenido una amplia acogida a la hora
 de evaluar características físicas junto con otros atributos como aspectos
 económicos y sociales.
 
\cite{rodman2006ageographic}  utilizan técnicas MCE para determinar posibles ubicaciones de turbinas
 en el norte de California evaluando componentes físicos,ambientales y humanos. \cite{janke2010multicriteria}
 categoriza diferentes aspectos de acuerdo al potencial eólico y solar e
 identifica áreas susceptibles a la instalación de turbinas y paneles solares
 en Colorado.
 
 Entre los aspectos evaluados se encuentran la distancia a carreteras y
 líneas de transmisión eléctrica, coberturas del terreno, densidad de población
 y áreas protegidas por la ley. \cite{petrov2014utilization} explora nuevos algoritmos de análisis de datos para modelar la ubicación
 de turbinas en Iowa apoyándose en un sistema espacial multi-criterio de
 soporte a la toma de decisiones.
 Nuevas técnicas de análisis de datos, comúnmente conocidas como minería
 de datos, han demostrado muy buenos resultados a la hora de modelar fenómenos
 atmosféricos.
 
 Por ejemplo, \cite{yusof2014miningfrequent} utiliza algoritmos para detectar patrones secuenciales en series de tiempo
 de viento desde estaciones en los Países Bajos para detectar anomalías
 en el flujo, velocidad y dirección del viento.

